\documentclass[12pt, a4paper, oneside]{scrartcl}

\usepackage{array}
\usepackage{version}
\usepackage{txfonts}
\usepackage{scrlayer-scrpage}
\usepackage{lastpage}
\usepackage{refcount}
\usepackage{todonotes}
\usepackage{fancyvrb}
\usepackage{pgf-umlsd}
\usepackage{pifont}
\usepackage[ngerman]{babel}
\usepackage[T1]{fontenc}

\foreach \i in {aufgabe1punkte, aufgabe2punkte, aufgabe3punkte, aufgabe4punkte, aufgabe5punkte, aufgabe6punkte, aufgabe7punkte, aufgabe8punkte, aufgabe9punkte} {
	\newcounter{\i}\setcounter{\i}{0}
}

\newcounter{sumGesamt}\setcounter{sumGesamt}{0}
\newcounter{nAufgabe}\setcounter{nAufgabe}{0}
\newcounter{sumNumText}
\newcounter{zwischSum}
\newcounter{sumhelp}

\newcommand{\xmark}{\ding{55}}
\renewcommand{\labelenumi}{(\alph{enumi})}
\setlength{\parindent}{0cm}

\newcommand{\sichermich}[2]{
	\begingroup
		\expandafter\def\csname the#1\endcsname{\number\csname c@#1\endcsname}
		\addtocounter{#1}{-1}\refstepcounter{#1}\label{#2}%
	\endgroup}

%% TEILAUFGABE-Umgebung%%%%%%%%%%%%%%%%%%%%%%%%%%%%%%%%%%%%%%%%%%%%%%%%%%%%%%%%%%%%%%%%%%%%%%%%%%%%%%%%%%%%%%%%%%%%%%%%%%%%%%%%%%%%%%%%%%%%%%%%%%%%%%%%%%%%%%%%%%%%%%%
\newenvironment{teilaufgabe}[1]
{  	
	\setcounter{sumGesamt}{\thesumGesamt+#1}
	\setcounter{zwischSum}{#1}

	\foreach \i/\j in {aufgabe1punkte/1, aufgabe2punkte/2, aufgabe3punkte/3, aufgabe4punkte/4, aufgabe5punkte/5, aufgabe6punkte/6, aufgabe7punkte/7, aufgabe8punkte/8, aufgabe9punkte/9} {
		\ifnum\thenAufgabe=\j
			\addtocounter{\i}{#1}
		\fi
	}
}
{
	\begin{flushright}
	\vspace{-0.7cm}
	\large
	\fbox{~~~~~~ / \thezwischSum}
	\vspace{-0.3cm}
	\end{flushright}
}

%% AUFGABE-Umgebung%%%%%%%%%%%%%%%%%%%%%%%%%%%%%%%%%%%%%%%%%%%%%%%%%%%%%%%%%%%%%%%%%%%%%%%%%%%%%%%%%%%%%%%%%%%%%%%%%%%%%%%%%%%%%%%%%%%%%%%%%%%%%%%%%%%%%%%%%%%%%%%%%
\newenvironment{aufgabe}[1]
{
	\setcounter{sumhelp}{#1}
	\setcounter{sumGesamt}{\thesumGesamt+\thesumhelp}

	\textbf{\sffamily Question \refstepcounter{nAufgabe}\arabic{nAufgabe}. }
	\ifthenelse{\value{nAufgabe}=1}{\newcommand{\sumNumText}{\ref{Punkte1}}}{}
	\ifthenelse{\value{nAufgabe}=2}{\newcommand{\sumNumText}{\ref{Punkte2}}}{}
	\ifthenelse{\value{nAufgabe}=3}{\newcommand{\sumNumText}{\ref{Punkte3}}}{}
	\ifthenelse{\value{nAufgabe}=4}{\newcommand{\sumNumText}{\ref{Punkte4}}}{}
	\ifthenelse{\value{nAufgabe}=5}{\newcommand{\sumNumText}{\ref{Punkte5}}}{}
	\ifthenelse{\value{nAufgabe}=6}{\newcommand{\sumNumText}{\ref{Punkte6}}}{}
	\ifthenelse{\value{nAufgabe}=7}{\newcommand{\sumNumText}{\ref{Punkte7}}}{}
	\ifthenelse{\value{nAufgabe}=8}{\newcommand{\sumNumText}{\ref{Punkte8}}}{}
	\ifthenelse{\value{nAufgabe}=9}{\newcommand{\sumNumText}{\ref{Punkte9}}}{}
	\hfill($\sumNumText$ Points)\\

	\begin{flushright}
		\vspace{-0.7cm}
		\Large
		\fbox{~~~~~~ / \sumNumText}
		\vspace{-0.3cm}
	\end{flushright}

	\foreach \i/\j in {aufgabe1punkte/1, aufgabe2punkte/2, aufgabe3punkte/3, aufgabe4punkte/4, aufgabe5punkte/5, aufgabe6punkte/6, aufgabe7punkte/7, aufgabe8punkte/8, aufgabe9punkte/9} {
		\ifnum\thenAufgabe=\j
			\setcounter{\i}{\thesumhelp}
		\fi
	}
}

