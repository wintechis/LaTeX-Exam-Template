% Alle nichtoptionalen Parameter der aufgabe-Umgebung sind Punktezahlen. 
% Sie müssen ganzzahlig sein, wenn Punktezählen aktiviert ist; Nullen werden ignoriert.
% Falls Punkte zählen deaktiviert ist, muss als letzter (8.) Parameter die Summe angegeben werden.
% Die [Kurzbeschreibung] ist optional, danach MÜSSEN 8 numerische Parameter kommen.
% Für die Bonusklausur werden die Parameter vollständig ignoriert.

\newpage
\begin{aufgabe}{0}{0}{0}{0}{0}{0}{0}{0}
\vspace{0.2cm}



\begin{enumerate}

	\begin{teilaufgabe}{3}
		\item Complete the following sequence diagram. 
		\vspace{0.5cm}
		\begin{lernziel}
			\textbf{G 1.5}
		\end{lernziel}
		
		\begin{nichtloesung}
			\begin{sequencediagram}
				\def\unitfactor{0.9}
				\newthread{useragent}{:\ User Agent}
				\newinst[7.5cm]{server1}{ex.org :\ Server}
				%
				%\begin{call}{useragent}{ GET /data\#Karlsruhe HTTP/1.1 \\ \_\_\_\_\_\_\_\_\_\_\_\_\_\_\_\_\_: Mozilla 5.0 (Chrome/49.0.2623.112)}{server1}{HTTP/1.1 404 \_\_\_\_\_\_\_\_\_\_\_\_\_\_\_\_\_}
				%\end{call}
				%
				%
				\begin{call}{useragent}{ \_\_\_\_\_ /mygeo HTTP/1.1 }
				%
				{server1}{HTTP/1.1 \_\_\_\_\_ Not found % \\ Content-Location: \_\_\_\_\_\_\_\_\_\_\_\_ 
				\\ \_\_\_\_\_\_\_\_\_\_\_\_\_\_\_: text/html; charset=utf-8
				}
				\end{call}
			\end{sequencediagram}
		\end{nichtloesung}

		\begin{loesung}
			\textbf{Lösung:}
			\begin{sequencediagram}
				\def\unitfactor{0.9}
				\newthread{useragent}{:\ User Agent}
				\newinst[7.5cm]{server1}{ex.org :\ Server}
				%
				%\begin{call}{useragent}{ GET /data\#Karlsruhe HTTP/1.1 \\ \_\_\_\_\_\_\_\_\_\_\_\_\_\_\_\_\_: Mozilla 5.0 (Chrome/49.0.2623.112)}{server1}{HTTP/1.1 404 \_\_\_\_\_\_\_\_\_\_\_\_\_\_\_\_\_}
				%\end{call}
				%
				%
				\begin{call}{useragent}{ GET /mygeo HTTP/1.1 }
				%
				{server1}{HTTP/1.1 404 Not found % \\ Content-Location: \_\_\_\_\_\_\_\_\_\_\_\_ 
				\\ Content-Type: text/html; charset=utf-8
				}
				\end{call}
			\end{sequencediagram}
		\end{loesung}
	\end{teilaufgabe}


	\begin{teilaufgabe}{2}
		\item Map the HTTP methods (DELETE, GET, POST, PUT) with their comparable CRUD Operation in the table below by completing the entries in the table. Note that there can be more than one value within a cell.
		\begin{lernziel}
		\textbf{G 4.1}
		\end{lernziel}
		\begin{nichtloesung}
			\begin{center}
						\setlength\extrarowheight{3pt}
				\begin{tabular}{|c|c|}
					\hline
					\textbf{CRUD Operations} & \textbf{HTTP Methods} \\ \hline
					Create                   &              \\ \hline
					                     	 & GET                   \\ \hline
					Update                   &                    \\ \hline
					Delete                   & 		                \\ \hline
					\end{tabular}
			  \end{center}
		\end{nichtloesung}
		
		\begin{loesung}
			\textbf{Lösung:}
			\begin{center}
				\begin{tabular}{|c|l|}
					\hline
					\textbf{CRUD-Operations} & \textbf{HTTP Methods} \\ \hline
					Create                   & POST, PUT             \\ \hline
					Read                     & GET                   \\ \hline
					Update                   & PUT                   \\ \hline
					Delete                   & DELETE                \\ \hline
					\end{tabular}
			  \end{center}
		\end{loesung}
	\end{teilaufgabe}


\end{enumerate}

\end{aufgabe}
