% Alle nichtoptionalen Parameter der aufgabe-Umgebung sind Punktezahlen. 
% Sie müssen ganzzahlig sein, wenn Punktezählen aktiviert ist; Nullen werden ignoriert.
% Falls Punkte zählen deaktiviert ist, muss als letzter (8.) Parameter die Summe angegeben werden.
% Die [Kurzbeschreibung] ist optional, danach MÜSSEN 8 numerische Parameter kommen.
% Für die Bonusklausur werden die Parameter vollständig ignoriert.

\newpage
\begin{aufgabe}{0}{0}{0}{0}{0}{0}{0}{0}
\vspace{0.2cm}
Task Question ...

\begin{enumerate}
\begin{teilaufgabe}{6}
\item
\begin{lernziel}
\textbf{G 3.2}
\end{lernziel}

In the following code blocks are visualized.
\begin{nichtloesung}
\begin{Verbatim}
  @prefix : <http://www.wiso-coffee.org/blog#> .
  @prefix rdf: <http://www.w3.org/1999/02/22-rdf-syntax-ns#> .
  @prefix rdfs: <http://www.w3.org/2000/01/rdf-schema#> .
  @prefix xsd: <http://www.w3.org/2001/XMLSchema#> .
  \end{Verbatim}
\vspace{8cm}
\end{nichtloesung}

\begin{loesung}
  \textbf{Lösung:}
  \begin{footnotesize}
    \begin{Verbatim}[frame=single]
      @prefix : <http://www.wiso-coffee.org/blog#> .
      @prefix rdf: <http://www.w3.org/1999/02/22-rdf-syntax-ns#> .
      @prefix rdfs: <http://www.w3.org/2000/01/rdf-schema#> .
      @prefix xsd: <http://www.w3.org/2001/XMLSchema#> .

      
      :bulldog      rdfs:label   "White Bulldog" ;
		  :offers	[   a    :americano;
				   :hasSize "normal";
                                             :hasPrice "3€" ],
                              		[   a    :americano;
				   :hasSize "big";
                                             :hasPrice "3.6€"^^xsd:string ].
      :americano   a                 Coffee;
		     rdfs:label    "Americano".
			


      
\end{Verbatim}
\end{footnotesize}  
\end{loesung}
\end{teilaufgabe}

\end{enumerate}
\end{aufgabe}