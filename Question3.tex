\newpage
\begin{question}{10}
	\vspace{0.2cm}
	
	\begin{enumerate}
		\begin{subquestion}{6}
			\item Complete the following sequence diagram. 
			\vspace{0.5cm}
			\begin{learningGoal}\textbf{G 1.5}\end{learningGoal}
			
			\begin{notSolution}
				\begin{sequencediagram}
					\def\unitfactor{0.9}
					\newthread{useragent}{:\ User Agent}
					\newinst[7.5cm]{server1}{ex.org :\ Server}
					\begin{call}{useragent}{ \_\_\_\_\_ /mygeo HTTP/1.1 }
					{server1}{HTTP/1.1 \_\_\_\_\_ Not found
					\\ \_\_\_\_\_\_\_\_\_\_\_\_\_\_\_: text/html; charset=utf-8
%% if there is a command to generate the blank line instead of writing \_ explicitly would be nicer. Use: \repeat{\_}{15}.
					}
					\end{call}
				\end{sequencediagram}
			\end{notSolution}
	
			\begin{solution}
				\textbf{Solution:}
				\begin{sequencediagram}
					\def\unitfactor{0.9}
					\newthread{useragent}{:\ User Agent}
					\newinst[7.5cm]{server1}{ex.org :\ Server}
					\begin{call}{useragent}{ GET /mygeo HTTP/1.1 }
					{server1}{HTTP/1.1 404 Not found
					\\ Content-Type: text/html; charset=utf-8
					}
					\end{call}
				\end{sequencediagram}
			\end{solution}
		\end{subquestion}
	
		\begin{subquestion}{2}
			\item Map the HTTP methods (DELETE, GET, POST, PUT) with their comparable CRUD Operation in the table below by completing the entries in the table. Note that there can be more than one value within a cell.
			\begin{learningGoal}\textbf{G 4.1}\end{learningGoal}
			\begin{notSolution}
				\begin{center}
					\setlength\extrarowheight{3pt}
					\begin{tabular}{|c|c|}
						\hline
						\textbf{CRUD Operations} & \textbf{HTTP Methods} \\ \hline
						Create                   &              \\ \hline
						                     	 & GET                   \\ \hline
						Update                   &                    \\ \hline
						Delete                   & 		                \\ \hline
						\end{tabular}
				  \end{center}
			\end{notSolution}
			
			\begin{solution}
				\textbf{Solution:}
				\begin{center}
					\begin{tabular}{|c|l|}
						\hline
						\textbf{CRUD-Operations} & \textbf{HTTP Methods} \\ \hline
						Create                   & POST, PUT             \\ \hline
						Read                     & GET                   \\ \hline
						Update                   & PUT                   \\ \hline
						Delete                   & DELETE                \\ \hline
						\end{tabular}
				  \end{center}
			\end{solution}
		\end{subquestion}
	\end{enumerate}
\end{question}
